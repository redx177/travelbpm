\chapter{Zusammenfassung \& Fazit}
Als erstes wurde eine Recherche betrieben, welches BPMN Produkt sich für den Einsatz in diesem Projekt eignet. Die Entscheidung viel auf die Software Bonita von der Firma Bonitasoft. Sie deckt den grössten funktionalen Umfang ab und hat eine ansprechende und funktionale Benutzeroberfläche. 

Anschliessend wurde eine Analyse gemacht, wobei die anzusprechende \Gls{glos:api} erläutert und die Funktionen von Bonita beschrieben wurde.

Beim darauffolgenden Konzept sind Mockups für die zu verwenden Formulare direkt in Bonita erstellt worden. Dies war sehr einfach da der UI Designer der Software sehr simple zu bedienen ist. Zusätzlich wurde der Prozess beschrieben, welcher in BPMN nachgebaut werden soll.

Die Umsetzung in Bonita startete sehr speditiv. Der Prozess wurde definiert, die Formulare angebunden sowie die Contracts und Operations hinterlegt. Danach wurde die Umsetzung umständlicher. Das erste Problem war, dass die Resultate der Connectors nicht ins Bonita überführt werden konnte. Es wurden stattdessen Standartwerte definiert, damit man den Prozess zumindest durchführen kann.

Danach offenbarte sich ein weiteres Problem mit den Connectors. Bei deren Aufruf gab es eine weitere Fehlermeldung im Log und der Prozess brach ab. Gelöst werden konnte dies nur dadurch, dass die Connectors ganz entfernt wurden.

Zusammenfassend lässt sich sagen, dass Bonita wohl die falsche Wahl war für dieses Projekt. Das Programm ist sehr ansprechend und es lässt sich sehr schnell ein Prozess damit definieren und ausführen. Jedoch war die Aufgabenstellung dieses Projektes nicht dazu geeignet, in Bonita umgesetzt zu werden.

Spass hat dieses Projekt trotzdem gemacht. Ich habe einiges zu BPMN gelernt und mit Groovy Einblicke in eine mir neue Programmiersprache erhalten. Schade ist nur, dass der Aufwand für die Entwicklung der Connectors im Endprodukt nicht eingeflossen ist.